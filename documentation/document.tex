\documentclass[a4paper]{article}
\usepackage[a4paper, margin=2.54cm]{geometry}
\usepackage[german]{babel}
\usepackage[UTF8]{inputenc}
\begin{document}

	\section{Einführung}
	\subsection{Motivation}
	In der heutigen Zeit wird der Verwendung von Mikrocontrollern eine immer höre Bedeutung zugemessen. Diese entsteht durch die einfache Programmierung auf Systemebene in Assebly sowie die konstengünstige und platzspaarende Produktion.\\
	Die Verwendung des 8051  bietet hierbei eine bekannte Plattform, welche sich für die Umsetzung einer Vielzahl an kreativen Projekten eignet.\\
	Die Idee dieses Projektes besteht darin, die Hardware mit verschiedener Peripherie zur Eingabe von Morse-Code zu nutzen.\\
	\subsection{Aufgabenstellung}
	Die Aufgabenstellung war es ein Programm für den 8051 Microcontroller zu schreiben, das folgende Anforderung erfüllt:
	\begin{itemize}
		\item Compilierfähigkeit des Programmes
		\item Verwendung eines Timers
		\item Verwendung eines Interrupt
	\end{itemize}
	Zur Vereinfachung wurde anstatt eines 8051 ein Simulationsprogramm verwendet, welche sowohl den 8051, als auch die Ein und Ausgabehardware simuliert.\\
	Der fertige Code ist offen auf Github zu finden.
	
	
	\section{Grundlagen}
	\subsection{Assembler}
	Als Assembler oder Assebly bezeichnet man eine Programmiersprache, mit welcher sich eine bestimmte Hardware oder Prozessorachitektur gezielt programmieren lässt. Asseblercode wird also gezielt für eine Architektur entwickelt und lässt sich nur auf dieser ausführen.\\
	Der Quellcode bildet hierbei eine Folge von Maschinenbefehlen. Maschinenbefehle bestehen dabei aus einem Operationscode und meist einer weiteren Folge von Angaben wie Adressen oder Literalen.\\
	\ref{ls:mov} zeigt hierbei einen einfachen MOV - Befehl in  der Maschinensprache von x86 Prozessoren. MOV bedeutet hierbei so viel wie mov-byte von/was, nach.\\
	
	\section{Konzept}
\end{document}


